\documentclass[a4paper,titlepage]{report}
\pagestyle{plain}
\usepackage[french]{babel}


\usepackage{tracklang}

\usepackage{latexsym}
\usepackage[T1]{fontenc}
\usepackage[utf8,latin1]{inputenc}

\usepackage[]{algorithm2e}
\usepackage{titlesec}
\usepackage{xspace}
\usepackage[version=1]{mhchem} % Package for chemical equation typesetting
\usepackage{graphicx}
\usepackage{amsmath} % Required for some math elements
\usepackage{amssymb}
\usepackage[babel=true]{csquotes}
\usepackage[backend=biber,bibencoding=utf8]{biblatex}
\usepackage[toc,lof,lot]{multitoc}
\renewcommand*{\multicolumntoc}{2}
\setlength{\columnseprule}{0.5pt}
\renewcommand*{\multicolumnlof}{2}
\renewcommand*{\multicolumnlot}{2}
% package tableaux
\usepackage{booktabs}
\usepackage[table,xcdraw]{xcolor}

% package glossaire
\usepackage[acronym]{glossaries}
\usepackage{xparse}

\RestyleAlgo{boxed}

\setlength\parindent{0pt} % Removes all indentation from paragraphs

\newcommand{\HRule}{\rule{\linewidth}{0.5mm}}
\newcommand{\Title}{Traitement Numérique de l'Image\xspace}
\renewcommand{\labelenumi}{\alph{enumi}.} % Make numbering in the enumerate environment by letter rather than number (e.g. section 6)



% bibliographie
\addbibresource{rapport.bib}
\DefineBibliographyStrings{english}{%
	bibliography = {Bibliographie},
}

% glossaire  
\makeglossaries
\setglossarystyle{altlist}

\DeclareDocumentCommand{\newdualentry}{ O{} O{} m m m m } {
	\newglossaryentry{gls-#3}{name={#5},text={#5\glsadd{#3}},
		description={#6},#1
	}
	\newacronym[see={[Glossary:]{gls-#3}},#2]{#3}{#4}{#5\glsadd{gls-#3}}
}


\newacronym{tic}{TIC}{Technologies de l'Information et de la Communication}
\newacronym{moe}{MOE}{maîtrise d'œuvre}
\newacronym{moa}{MOA}{maîtrise d'ouvrage}

\newglossaryentry{snap}
{
	name=Snapshot,
	description={Type de copie de volume se réalisant de façon quasi instantanée et consommant peu d'espace disques grâce à l'utilisation d'un mécanisme de pointeurs entre les volumes sources et les volumes cibles.}
}

\title{Traitement Numérique de l'Image}

\author{Terry \textsc{DERVAUX}}
\date{\today} % Date for the report

\titleformat{\chapter}
{\Large\bfseries}	% Format
{}					% Label
{0pt}				% Séparation
{\huge}				% Before code

\begin{document}
	\begin{titlepage}
		\center
		\includegraphics[width = 100mm]{./images/logo_telecom_nancy.png}
		\HRule \\[0.4cm]
		{\huge \bfseries \Title}\\[0.4cm]
		{ \bfseries Compte rendu TP3 et TP4}\\[0.4cm]
		\HRule \\[1.5cm]
		
		\begin{minipage}[t]{0.45\textwidth}
			\begin{flushleft} \large
				\emph{Auteurs:}\\
				Terry \textsc{Dervaux}\\
				Anthony \textsc{Fioravanti}\\
			\end{flushleft}
		\end{minipage}
		~
		\begin{minipage}[t]{0.45\textwidth}
			
			\begin{flushright} \large
				\emph{Enseignant :}\\
				Vincent \textsc{Bombardier}\\
			\end{flushright}
			
			
		\end{minipage}\\[1.5cm]
		\HRule \\[1.5cm]
		
		Date: \today\\
		
		\vfill
	\end{titlepage}
	
	\tableofcontents
	
	\chapter{Introduction}
		\paragraph{}
		Dans cette partie, je vais vous situer rapidement le contexte de mon projet d'approfondissement et vous présenter la mission qui m'a été confiée.
		\section{Contexte}
		
	
	\listoffigures	
	
	\printglossaries
	
	
	\printbibliography
	
\end{document}
